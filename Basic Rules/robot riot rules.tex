\documentclass{article}
\title{Robot Riot Ruleset v1.0}
\author{Adrian Choy}
\date{March 2017}
\begin{document}
   \maketitle
   \section {Background}
   	The purpose of this document is to outline Robot Riot, an amateur robotics competition. Robot Riot was designed in response to the restrictive and exclusive nature of competitive robotics. Many competitions, such as FIRST or Battlebots, require near-expert knowledge, mentorship, impossible amounts of funding and equipment that scare away many would-be amateur entrants. In addition many competitions are stagnant and result in min/maxing where variety and creativity are punished. This is either due to a static rule set or the very nature of the type of competition involved. As a result robot competitions tend to be no more than homogenous groups of experienced engineers, entertaining only themselves and doing little to encourage outsiders to participate.
	
	Robot Riot has been designed to address these flaws. Participants will have to both show off their building skills and win the audience's favor in order to pass each round. Each competition will also have its own themes and adjusted rulesets, forcing participants to stay on their toes. Additionally the spirit of Robot Riot will encourage spectacles and theatrics. The competition will boast all the rivalries, trash talking, and sensationalism found in sports both traditional and nontraditional. The competition has been designed to promote STEAM skill development alongside creative bravado in order to draw in competitors from all skill levels and audiences from all backgrounds.
	
	If the DARPA Challenges can be considered the Olympics and Battlebots can be considered boxing then ultimately I want people to consider Robot Riot as Wrestlemania.
   
   \section {Game Description}
   	
	This section will describe any relevant logistics related directly to the game itself.
	 	
	   \subsection {Participant/Organizer Agreements}
	   	
	   	The following items will be consolidated in a registration form for participation. A participant may not enter unless they agree to the following terms.
		
		\begin{itemize}
  			\item  The organizers are not responsible for damages done to participant robots at any point prior, during, or after the game. Participants will enter with the full understanding that their robots and equipment are subject to destruction.
 			\item Participants will agree that photo and video may be taken of them and used to promote the competition.
  			\item Participants will not engage in behavior that results in willful destruction and disrespect of venue, sponsor, or competition property.
 			\item Participants who engage in extreme cases of poor sportsmanship and violation of the Participant Code of Conduct (Section 2.3) are subject to immediate removal from the competition and venue.
  			\item The organizers retain the right to disqualify participants at any time if they feel that a participant breaches the rules or otherwise hinders the spirit of the competition. They also retain the right to disqualify a participant if it is felt that their robot poses a threat to human safety or has the potential to damage the venue's property.
		\end{itemize}


	   
	   \subsection {Qualifications}
	   
	   	The following conditions must be met for participants to be eligible for competition. Participants must submit a form and a video of their robot no later than a week before the competition to be officially registered.
		
		The qualifications have been designed to be lax enough to allow participants to use their full creativity in designing and piloting their robots while also ensuring that participants keep true to the spirit of the competition.
	   	
		
	  	 \subsubsection {Participant Qualifications}	
	   
		\begin{itemize}
  			\item Participants must be 21 or over. A separate competition may be available for potential participants under 21 in the future given Robot Riot?s demand and popularity.
 			\item Participants can enter as an individual or within a group.
  			\item  Each individual or group can enter as many robots as they?d like, though each robot must have a different pilot and only one robot may be entered in a single bracket slot.
 			\item Participants will be given a form that they will be required to sign and submit before the
competition. The form allows the event organizers to use photo and video of participants and their robots, details rules and responsibilities, and outlines required and optional behavior at the event.
  			\item Participants must submit a video of the robot and all of its functions to the organizers before the day of the competition.
		\end{itemize}

	   \subsubsection {Personas}	
			Participants are HIGHLY encouraged to adapt personas, similar to those of pro wrestlers. These personas will be stage identities that participants take on during the competition. These personas will influence how participants talk, behave, and dress at competitions. For example for Super Robot Riot II Turbo the team from Trossen Robotics came dressed as Rick and Morty and entered a robot that looked like one from the TV show.
			
			Feel free to work with other participants in crafting personas and even going as far as to forming fictional rivalries. We encourage participants to use the competition as a creative outlet. Tell a story withyourpersonaandletitevolvethemoreyoucompete. Besuretomodifyyourpersonaifa particular competition has a theme!

	

	   \subsubsection {Participant Do's and Don'ts}	
	   	This section outlines recommended and discouraged participant behavior. Though ignoring this list will not result in disqualification it does guide participants towards preparing themselves in a way that is in the spirit of the competition. By following these guidelines a participant may be able to easily build a robot that can both be effective in combat while earning the favor of the crowd.
		\begin{itemize}
  			\item DO share ideas and suggestions with competition organizers.
 			\item DO conspire with competition organizers and other participants for theatrics. The bigger the
spectacle the better!
  			\item DO invite as many friends as possible to cheer for you during judgments. Yes, it may seem
like the judgment can devolve into a popularity contest but it is the sole responsibility of the
participants to sway the hearts of the people.
 			\item DO work with other participants to improve each other's robots and possibly form teams for
future competitions.
  			\item DO prepare a brief eulogy for your robot in the event that you enter a grudge match and
lose or your robot is otherwise destroyed in a normal match.
 			\item DO dress up for each competition.
  			\item DO engage in good sportsmanship
		\end{itemize}
		
		
				
		
		\begin{itemize}
  			\item DON'T be salty. No one likes a sore loser. Extremely poor sportsmanship will affect a participant's future in the competition.
 			\item DON'T push personal boundaries. "No" means no, "stop" means stop. Participants are expected to be able to discern the boundaries between being playful and being offensive if they choose to .
  			\item DON'T sweat it. 		
		\end{itemize}


	\subsection {Participant Code of Conduct}	
	
	These guidelines detail the expected behaviors of each participant. Breaching these guidelines may be grounds for disqualification.

	
	\begin{itemize}
  			\item Participants will respect the venue, judges, organizers, and any other parties that otherwise support or contribute to the competition.
 			\item Participants are expected to be competitive and boastful in a playful manner while maintaining an air of positive sportsmanship. "Trash talking" is encouraged but participants are expected to understand boundaries and limits.
  			\item Participants are encouraged to act and dress in eccentric personas, as in the spirit of professional wrestling and other similar acts.
 			\item Encouragement of bravado and theatrics is not an invitation for harassment. Robot Riot has adopted Geek Bar's Anti-Harassment Policy, seen below:
  	
		\end{itemize}
		
		Geek Bar Chicago is dedicated to providing a harassment-free experience for everyone, regardless of
gender, gender identity and expression, sexual orientation, disability, physical appearance, body size,
race, or religion. We do not tolerate harassment of guests, either in-person or online, in any form. Sexual
language and imagery are not appropriate at Geek Bar Chicago, except in select pre-scheduled events.
Guests and users violating these rules may be sanctioned or expelled from Geek Bar Chicago and offsite
Geek Bar experiences at the discretion of management. Harassment includes offensive verbal or textual
comments related to gender, gender identity and expression, sexual orientation, disability, physical
appearance, body size, race, or religion. Harassment also includes posting sexual images, deliberate
intimidation, stalking, following, harassing photography or recording, sustained disruption of
conversations, talks or other events, inappropriate physical contact, and unwelcome sexual attention.
Guests and users asked to stop any harassing behavior are expected to comply immediately. If a guest or
user engages in harassing behavior, management may take any action they deem appropriate, including
warning the offender, banning the offender or expelling the offender from Geek Bar and offsite Geek Bar
experiences. If you are being harassed, notice that someone else is being harassed, or have any other
concerns, please contact a member of Geek Bar staff immediately. Geek Bar staff will be happy to help
participants contact local law enforcement or otherwise assist those experiencing harassment to feel
safe. We value your patronage. We expect guests and users to follow these rules at all times.
		
Participants are expected to act within the guidelines of this policy. Violators of this policy will receive non-conditional permanent ban from the competition.
	
	\subsection {Game Rules}	
		\subsubsection {Bracket}	
			
                        The matches will be organized in a standard single-elimination bracket. Organizing participants into teams or assigning them labels (if applicable) will occur before assigning bracket slots. These slots on the bracket will be assigned randomly in a way that avoids pairing participants in like teams/labels.
			
                        Competitions will feature a minimum of an 8 robot bracket. The competition may be increased to a 16 robot bracket and/or introduce wildcard slots that are earned via free for all matches depending on how many participants apply. Prior to inviting applicants to participate the organizers will choose a number of slots to reserve as wildcard slots. The most promising participants will be promised a slot in the bracket. The remaining invited participants will participate in at least one free for all match where more than two robots compete simultaneously. There will be one free for all match for each wildcard slot which will be awarded to the robot that wins based on the win conditions outlined in section 2.4.3.
			
                        Slots may be reserved for recipients of the Veterans Seat award as listed in section 2.6. In these instances a slot will be reserved for a previous competition's award recipient. However if an award recipient declines their slot it will be offered to another non-recipient participant.
	
		\subsubsection {Match Proceedings}	
		
                		The bracket will be posted publicly once all the slots have been assigned. There will be a brief opening overview where an MC will announce the rules of the competition and the prizes that await the winners. This MC will preside over all remaining announcements for the competition. Participants are expected to arrive near the stage during the match that takes place prior to theirs. Introductions for each participant will be made before each participant's first match in the competition. Participants will also be asked if they would like to turn the match into a Grudge Match, whereupon the loser must destroy their own robot at the end of the match.
		
                		Each match will last no longer than two minutes. Matches will be initiated by a countdown and a time limit warning will be announced 30 seconds before the end of each match. Each robot will start at any desired position within each participant's corresponding half of the arena. Matches will end early if a clear winner is decided beforehand or if the judges deem that the match must end early. Judgment will occur at the end of each match and a winner will be decided as outlined in section 2.4.3. Judging, conducted by the referee, will occur after each match assuming a clear winner hasn't been decided.
		
              		  In the event that the competition is predicted to last longer than the venue allows any halftime shows will be cancelled or the final rounds will be a free-for-all battle where all the remaining robots will compete in one final 5 minute round.
                
			Each phase and its estimated time length are listed in the table below. A schedule of events will be created before each competition based on the number of anticipated participants.
                	
\begin{center}
    \begin{tabular}{| l | l | l | l |}
    \hline
    Phase & Time \\ \hline
    Introduction &  10 inutes \\ \hline
    Wildcard Qualifier & 5 Minutes\\ \hline
    Match Setup & 5 Minutes \\ \hline
    Participant Intro & 5 Minutes  \\    \hline
    Match &  2 Minutes \\    \hline
    Judging &  3 Minute \\    \hline
    Break &  TBD \\    \hline
    Closing and Awards &  10 Minutes \\    \hline
    \end{tabular}
\end{center}



		\subsubsection {Win Conditions}	
				
			At the end of each match a winner will be decided based on the following win conditions. They take priority in the following order.
				
			\begin{enumerate}
                          \item A participant decides to forfeit the match for any reason.
                          \item A participant becomes disqualified.
                          \item A participant is destroyed beyond repair.
                          \item A winner is declared by the end-of-match pollice verso. The pollice verso, roughly translated to
"with a turned thumb", is a Romanesque crowd vote. The names of each robot will be announced in turn. The robot with the most cheers following their announcement will be declared the true winner. The MC will be tasked with deciding which participant gained the most cheers though the judges will contribute to this decision as seen necessary.

                        \end{enumerate}


		\subsubsection {Terms for Disqualification}	
				A participant can be disqualified for the following reasons:
				
				
	\begin{itemize}
  			\item The participant fails to arrive a reasonable time before their match.
 			\item The participant modified their robot significantly between their check-in and their match in such a way that breaches the robot qualification guidelines.
  			\item The referee decides the participant should be removed from competition for any reason.
 			\item The participant breaches the Participant Code of Conduct in a severe enough manner to warrant
removal from the competition.
  			\item The participant damaged another participant's robot, intentionally or unintentionally, outside of
designated matches.
		\end{itemize}





	\subsection {Arena Description}	
            		
            The arena, nicknamed "Lil Colliseo", will be made of wood. The arena will be an elevated cylinder of 3' diameter and 3.5' height. The ring boundaries will match the perimeter of the circle. Boundary lines will be painted as a white 1" border around the circle's perimeter. Each half of the circle will be painted red and blue respectively with the halves divided by a 1" thick white line
		.
            The arena will be quickly cleaned after each match. Permanent damage to the arena caused by a match will be left alone through the duration of the competition.
            		

	\subsection {Awards}	

	\subsection {Win Conditions}	
	
	Awards may be distributed by either the competition organizers, the venue, or any competition partners. Awards are distributed in the following categories:
	
	
\begin{center}
    \begin{tabular}{| l | l | l | l |}
    \hline
    Award & Condition & Prize  \\ \hline
    Champion &  First place in Bracket & Trophy \\ \hline
    Veterans Seat & Place in Top 4 of Brackets & Guaranteed Slots in Next Competition \\ \hline
    First Place in Design & Voted "First in Show" by Audience & Ribbon  \\ \hline
    Worst Place in Design & Voted ?Worst in Show? by Audience  &  Ribbon \\    \hline
    \end{tabular}
\end{center}

The Veterans Seat award rewards skillful participants and encourages them to return to future competitions. The First Place in Design award will be awarded to a robot voted by the audience as the best in show in terms of aesthetics and showmanship. The Worst Place in Design is also crowdvoted but in terms of the worst-designed robot. The voting for awards will occur before the competition. Before the competition each robot will be displayed with a bucket next to each one. Guests will receive two tickets at the door and will drop their ticket in their desired robot?s bucket for each prize

Additional awards and prizes may be given at the discretion of sponsors and venue providers. Competition organizers will generally recommend venue providers to offer same-night bar tabs for top placing participants. Venues will also be encouraged to assign drinks or menu items with each participant and offer a discount for attendees on certain items if the matching participant wins the competition. Other suggestions include gift cards and event tickets.

	
\section {Robot Design Guide}	

This section will outline suggested robot design methodologies for participants. These methodologies outline suggested materials and behaviors for your robots that are in the spirit of the competition, though you are free to go your own route within the restraints of the robot qualification guidelines (Section 3.1). This competition is very strange in the sense that "winning" will have a different meaning for each participant. To some this will mean making a robot that can push and destroy other robots. To others this will mean making a robot that is entertaining, novel, and fun to watch.

	\subsection {Robot Qualifications}	
				
				The qualifications have been designed to be lax enough to allow participants to use their full creativity in designing and piloting their robots while also ensuring that participants keep true to the spirit of the competition.
				
	\begin{itemize}
  			\item  Robots must have physical dimensions that allow it to fit within their respective half-circle at the start of the match without extending over the edges or into their opponents half-circle. A description of the arena can be found in section 2.5.
 			\item A robot is allowed to separate into multiple robots but each robot must either be autonomous or controlled by a single pilot.
  			\item Robots must be designed to be mobile.
 			\item Robots may have physical projectiles but fire projectiles are forbidden. Physical projectiles may
not be comprised of or contain sharp objects.
  			\item Robots may not be designed to intentionally use corrosives in any way
 			\item Robots must not be designed to intentionally damage the playing field.
		\end{itemize}
		
		The organizers retain the right to disqualify any robot that fits these criteria if they feel the robot goes against the spirit of the competition or pose a potential safety threat.
				
				
	\subsection {Robot Do's and Don'ts}	
			
		This section outlines recommended and discouraged robot design methodologies. Though ignoring this list will not result in disqualification it does guide participants towards designing robots that are in the spirit of the competition. By following these guidelines a participant may be able to easily build a robot that can both be effective in combat while earning the favor of the crowd.
Do's
		
	\begin{itemize}
 			\item DO add dances, taunts, flashing lights, and playable songs to your robot.
  			\item DO consider different methods of control. Remote control, random movement, or even
autonomous sensing/planning/acting systems are game. Razzle dazzle the crowd with technical prowess or entertain the crowd and bewilder your opponent with erratic, unpredictable plays!
 			\item DO design your robot with signature moves. Remember, this is the Wrestlemania of robotics. Give your robot the mechanical equivalent of the People's Elbow. Be sure to let the competition organizers know about these signature moves when you submit your form!
  			\item DO spend effort on how your robot looks. If you're cover is well-judged by your crowd it could be a surefire way to win the People's Choice award.
 			\item DO be flexible with your robot. Make it easy to modify, even consider adding parts that can easily be removed and added to other competing robots! You may change or make additions to your robot during the competition itself given that these alterations do not go against the other mandatory guidelines. You may even want to cooperate with other participants to swap parts. In similar competitions such as Hebocon it's encouraged for losing participants to offer parts of their robots to the robots who defeated them so their fighting spirit can live on.
  			\item DO be conscious of a particular competition's theme while designing robots. Some competitions may have themes such as "pro wrestling" or "Mad Max". You may want to design both a robot's look and functions to match the theme. Use the competition's atmosphere to your advantage!
 			\item DO make a robot modular! If it's lucky enough to survive the first competition it enters you can easily modify it for future fights. It's survival of the fittest! Though I guess when we're talking about robots that can't reproduce that may not be the best use of the colloquialism. I'm sure you understand what I'm getting at anyway.				
	
		\end{itemize}


	
	Dont's
				
	\begin{itemize}
  			\item DON'T design your robot so that it poses a safety threat. If you have to ask yourself "is this safe?" it probably isn't.
 			\item DON'T go for the easy win. Maximizing kill efficiency may not necessarily help you win the judgment vote.
  			\item DON'T add anything to your robot that you will miss. Be at peace with the knowledge that your robot will possibly be subject to complete and utter destruction. Victory, defeat, material desire; it's all fleeting! Live for the moment!
		\end{itemize}


	
				
				
	\begin{itemize}
  			\item 
 			\item
  			\item 
 			\item
  			\item 
 			\item
  			\item 
 			\item
  			\item 
 			\item
  			\item 
 			\item
  			\item 
 			\item
		\end{itemize}



   
\end{document}